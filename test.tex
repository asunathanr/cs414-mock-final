% !TEX TS-program = pdflatex
% !TEX encoding = UTF-8 Unicode

% This is a simple template for a LaTeX document using the "article" class.
% See "book", "report", "letter" for other types of document.

\documentclass{exam} % use larger type; default would be 10pt

%%% END Article customizations

%%% The "real" document content comes below...

\usepackage{syntax}

\newcommand{\spacedQuestion}[2][1] {
  \question
  {#2}
  \vspace{\stretch{#1}}
}

\begin{document}

\begin{questions}

  \spacedQuestion {
    Why do we study programming languages? (Give three reasons)
  }


  \spacedQuestion {
    One day your project manager decides that every new program your team produces is to be written in machine language, since computers don't understand C++ anyway. Other than referring them to a therapist, what are some things you could tell them to alleviate the situation?
  }

  \spacedQuestion { Define programming language: }

  \spacedQuestion { What is the fetch-execute cycle? }

  \spacedQuestion { What was the first high level language? }

  \spacedQuestion { Who invented Fortran? }

  \spacedQuestion { When was Fortran invented? }

  \spacedQuestion { What is the only high level structure in Fortran? }

  \spacedQuestion { What was the purpose of the do-loop with regards to efficiency? }

  \spacedQuestion { Define a compiler }

  \pagebreak

  \question Define an Interpreter:
  \vspace{\stretch{1}}

  \question How does a compiler differ from an interpreter?
  \vspace{\stretch{1}}

  \question What did the loader do in the first assignment (psuedo-code interpreter)?
  \vspace{\stretch{1}}

  \question Explain how psuedo-code interpreters are related to virtual computers.
  \vspace{\stretch{1}}

  \question How many IO statements did Fortran have?
  \vspace{\stretch{1}}

  \question How was floating point calculation handled in the earliest computer systems?
  \vspace{\stretch{1}}

  \question Name each advance related to the IBM 704 as discussed in class.
  \vspace{\stretch{1}}

  \question How were Fortran subprograms compiled with respect to the entire program?
  \vspace{\stretch{1}}

  \question Explain how Fortran implemented recursion.
  \vspace{\stretch{1}}

  \question What is static allocation?
  \vspace{\stretch{1}}

  \question Why are GOTO statements considered harmful?
  \vspace{\stretch{1}}

  \question Define the Static Structure Principle
  \vspace{\stretch{1}}

  \question Define the Zero, One, Infinity Principle
  \vspace{\stretch{1}}

  \question Derive the Fortran 1-D addressing equation
  \vspace{\stretch{1}}

  \question Derive a 2-D addressing equation from the previously derived equation.
  \vspace{\stretch{1}}

  \question What is the origin of Fortran control structures?
  \vspace{\stretch{1}}

  \pagebreak

  \question FORTRAN Syntax/Semantics
  \begin{parts}
    \part Do loop
    \vspace{\stretch{1}}

    \part Arithmitic IF
    \vspace{\stretch{1}}

    \part GOTO statement
    \vspace{\stretch{1}}

    \part Common block
    \vspace{\stretch{1}}

    \part EQUIVALENCE statement
    \vspace{\stretch{1}}
  \end{parts}
  \vspace{\stretch{1}}
  \pagebreak

  \question ALGOL Statistics
  \begin{parts}
    \part Year?
    \vspace{\stretch{1}}
    \part Who invented it?
    \vspace{\stretch{1}}
    \part Name three objectives/goals for the newly created ALGOL.
    \vspace{\stretch{1}}
    \part Explain the main difference between ALGOL-60 and ALGOL-58.
    \vspace{\stretch{1}}
    \part Practically nobody wrote programs for ALGOL and its definitly not used today. Why is it such a big deal?
    \vspace{\stretch{1}}
    \part Why did nobody write programs for ALGOL?
  \end{parts}
  \vspace{\stretch{1}}
  \pagebreak
  
  \question BNF
  \begin{parts}
    \part What does BNF stand for? (hint: its not Best New Friend)
    \vspace{\stretch{1}}

    \part Who invented BNF
    \vspace{\stretch{1}}

    \part Where was it first used?
    \vspace{\stretch{1}}

    \part What is the purpose of BNF?
    \vspace{\stretch{1}}
  \end{parts}
  \pagebreak

  \question Define variable scope within the context of programming languages
  \vspace{\stretch{1}}

  \question Explain the dangling-else problem.
  \vspace{\stretch{1}}

  \question How long does each of the below scopes bind a variable to an address?
  \begin{parts}
    \part Global
    \vspace{\stretch{1}}

    \part Function
    \vspace{\stretch{1}}

    \part Block
    \vspace{\stretch{1}}
  \end{parts}

  \question What is the name of the area in memory that stores variables?
  \vspace{\stretch{1}}

  \question Isn't a compound statement just a fancy word for a block? Why/Why not? What is a block anyway?
  \vspace{\stretch{1}}

  \pagebreak

  \question Define feature interaction in the context of programming languages.
  \vspace{\stretch{1}}


  \question Name one feature interaction in FORTRAN.
  \vspace{\stretch{1}}

  \question What is dynamic scoping?
  \vspace{\stretch{1}}

  \question Define Grammar in the context of programming languages.
  \vspace{\stretch{1}}

  \question How do we define a Grammar for a programming language?
  \vspace{\stretch{1}}

  \question What is the syntax of the rewrite rule?
  \vspace{\stretch{1}}

  \question {Produce a parse tree for the statement `b c f d f' with the below grammar using a top down technique:
    \begin{grammar}
      <S> $\rightarrow$ b <A> <S> | <B>

      <A> $\rightarrow$ <A> <B> d | c

      <B> $\rightarrow$ f
    \end{grammar}
  }
  \vspace{\stretch{1}}

  \question Define parsing
  \vspace{\stretch{1}}

  \question List Chomsky's four types of grammar
  \vspace{\stretch{1}}

  \question Which types can today's programming languages parse?
  \vspace{\stretch{1}}

  \question Draw a link list structure for the following LISP code \\ (car (cons '(a b c) (cdr (cons '(a b) '(c d)))))

  \question What would be the results of the following LISP commands?
  \begin{parts}
    \part (cons '(too) '(be or not (to be)))
  \end{parts}

  \question What ALGOL-60 feature does Jensen's device rely on?

  \question Pascal Statistics
  \begin{parts}
    \part Year?
    \part Who invented it?
    \part What was Pascals contribution to programming languages?
  \end{parts}

\end{questions}

\end{document}